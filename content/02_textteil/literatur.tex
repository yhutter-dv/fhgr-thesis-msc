\chapter{Literaturübersicht}
\label{kap:literaturübersicht}
Die nachfolgende Literaturübersicht bietet einen Überblick über relevante Projekte und Visualisierungen, welche sich mit Zugverspätungen und deren Auswirkungen auf ein Zugnetzwerk befassen. Nebst den visuellen werden auch die algorithmischen Verfahren, insbesondere zur Erkennung von Mustern, betrachtet.

\section{Visualisierungs-Pipeline}
\textbf{Dashboards} sind ein effektives Werkzeug, um grosse Mengen an Daten verständlich und auf einen Blick darzustellen. Um ein qualitativ hochwertiges Dashboard zu erstellen, müssen sowohl die Daten und Algorithmen als auch die Visualisierungen und Interaktionen aufeinander abgestimmt sein. Um das Zusammenspiel dieser Komponenten zu veranschaulichen, werden häufig \textbf{Visualisierungs-Pipelines} verwendet (siehe Abbildung \ref{fig_pipeline_traffic_visualization}).

\begin{figure}[H]
    \caption{Visualisierungspipeline für Verkehrsdaten \parencite[S. 2971]{survey_traffic_data_visualization_2015}}
    \includegraphics[width=.8\linewidth]{content/00_assets/traffic_visualization_pipeline.png}
    \label{fig_pipeline_traffic_visualization}
\end{figure}

Das Paper von Chen et al. betrachtet verschiedene Visualisierungen zum Thema Verkehrsdaten anhand einer solchen Pipeline. Eine Visualisierung über Verkehrsdaten durchläuft gemäss Chen et al. vier verschiedene Phasen. In der Phase \textit{Raw Data} werden die Daten erfasst. Dies geschieht meist über Sensoren, GPS oder Log-Aufzeichnungen der Geräte. Anschliessend werden die Daten in der \textit{Data Preprocessing} Phase aufbereitet. Im Rahmen des Data Preprocessing werden zudem Datenfehler und Ausreisser entfernt. Aufgrund der grossen Datenmengen, welche bei Verkehrsdaten auftreten, ist es wichtig, dass diese Daten effizient gespeichert werden. Dies geschieht über \textit{Datenaggregationen} und die richtige Wahl eines effizienten \textit{Datenverwaltungssystems}. Das Ergebnis des Data Preprocessing ist ein Datensatz mit sowohl räumlichen, zeitlichen als auch multivariaten Informationen. Durch \textit{Visual Symbols} werden die Daten dem Nutzer mithilfe von unterschiedlichen Visualisierungen (bspw. Liniendiagramm) dargestellt. Um das Verständnis der Daten noch weiter zu fördern, werden im Schritt \textit{Visual Mapping} visuelle Variablen verwendet \parencite[S.2971]{survey_traffic_data_visualization_2015}. Ein Beispiel für die Verwendung von visuellen Variablen ist das Projekt \textit{Trains in Time} (siehe Abbildung \ref{fig_trains_in_time}). Hier werden die visuellen Variablen Grösse und Farbe verwendet, um die Zugverspätung und Zugauslastung im Pariser Streckennetz zu visualisieren. Die Farbe repräsentiert hierbei die Verspätung und die Grösse der Kreise die Auslastung eines Zuges \parencite{trains_of_data_2012}.   

\begin{figure}[H]
    \caption{Trains in Time \parencite{trains_of_data_2012}}
    \includegraphics[width=.5\linewidth]{content/00_assets/trains_in_time.jpg}
    \label{fig_trains_in_time}
\end{figure}

Ein weiteres Beispiel für die Verwendung von visuellen Variablen ist das Dashboard von Dimanche et al. (siehe Abbildung \ref{fig_massive_railway_visualization}). Dieses Dashboard richtet sich primär an Experten und erlaubt es, das Pariser Zugverkehrsnetzwerk zu explorieren. Die Züge werden hierbei als Rechtecke visualisiert. Die Breite des Rechtecks entspricht der Fahrzeit, die Länge entspricht der zurückgelegten Strecke und die Farbe widerspiegelt die Verspätung \parencite{visualization_tool_operating_experts_2017}.

\begin{figure}[H]
    \caption{Visualisierung von einzelnen Zügen und deren Verspätungen \parencite[S. 15843]{visualization_tool_operating_experts_2017}}
    \includegraphics[width=.5\linewidth]{content/00_assets/massive_railway_visualization.png}
    \label{fig_massive_railway_visualization}
\end{figure}

Die Visualisierungs-Pipeline ist ein wiederkehrender Zyklus, worin der Nutzer mithilfe von Interaktionen Einfluss auf diverse Aspekte dieses Zyklus nehmen kann. Durch Filtereinstellungen wird die Datenmenge beeinflusst. Mithilfe von Parametern kann direkter Einfluss auf die Funktionsweise der dahinterliegenden Algorithmen genommen werden\parencite[S.2971]{survey_traffic_data_visualization_2015}.

\section{Dashboard Visualisierungen}
Wie bereits angesprochen sind Dashboards eine beliebte Option, um komplexe und grosse Datenmengen zu visualisieren. Ein Dashboard erlaubt es, unterschiedliche Sichten auf den gleichen Datensatz zu erhalten. Dies geschieht mithilfe von unterschiedlichen Visualisierungen. Jede Visualisierung veranschaulicht hierbei einen anderen Aspekt der Daten. Visualisierungen sind keine Momentaufnahme, sondern interaktive Sichten auf Daten. Der Nutzer kann mittels diverser Einstellungen (Filtereinstellungen, Parameter etc.) Einfluss auf das finale Aussehen einer Visualisierung nehmen. Auch können die Visualisierungen mithilfe von Brushing und Linking Verfahren «verbunden» werden. 

Ein Beispiel für ein Dashboard, welches unterschiedliche Sichten ermöglicht, ist das Dashboard von Jeph et al. (siehe Abbildung \ref{fig_raildash}). Ihr Dashboard visualisiert den Einfluss von Ereignissen auf den Zugverkehr in Tokyo sowie auf das Laufverhalten von einzelnen Personen. Hierbei wurden grosse Mengen von Smartphone-GPS-Daten, Daten über Ereignisse und Events, sowie Zugnetzwerkdaten kombiniert \parencite{raildash_2022}.

\begin{figure}[H]
    \caption{Raildash Dashboard \parencite[S. 93]{raildash_2022}}
    \includegraphics[width=.5\linewidth]{content/00_assets/raildash.png}
    \label{fig_raildash}
\end{figure}

Um ein qualitativ hochwertiges Dashboard zu gestalten, gibt es diverse etablierte Patterns und Regeln. Zu den bekanntesten Regeln gehören die «8 goldenden Regeln» von Ben Shneidermann \parencite{golden_rules_dashboard}. Weitere zentrale Kriterien sind die Datenqualität und insbesondere die korrekte Wahl der passenden Visualisierung. Wird eine schlechte Visualisierung gewählt, resultiert dies in einem schlechten Dashboard. Es folgt daher ein Überblick über etablierte Visualisierungen und deren Anwendungsfälle, aufgezeigt anhand der zwei wichtigsten Datentypen in Verkehrsdaten, \textbf{räumliche} sowie \textbf{zeitliche} Daten. 

\subsection{Zeitliche Visualisierungen}
Bei den \textit{zeitlichen} Daten wird gemäss Chen et al. zwischen \textbf{linearen} und \textbf{periodischen} Zeitintervallen unterschieden. \textit{Lineare Zeitintervalle} haben einen definierten Start und Endpunkt. Solche Zeitintervalle beschreiben mittels Höhen- und Tiefpunkte, wie sich die Daten über die Zeit ändern. Für lineare Zeitintervalle werden häufig Liniendiagramme verwendet. Liniendiagramme sind einfach zu interpretieren, jedoch nicht die geeignete Wahl, wenn es um die Darstellung von \textit{mehreren Variablen} geht (Visual Clutter Problematik). Nebst Liniendiagramme bieten sich auch Theme Rivers an \parencite[S. 2973]{survey_traffic_data_visualization_2015}.

\textbf{Periodische Zeitintervalle} beschreiben wiederkehrende Prozesse mit einem bestimmten Intervall (Wochentage, Monate etc.). Für solche Daten eignen sich besonders radiale Visualisierungen. In Abbildung \ref{fig_radial_layout} wird ein Tag als Kreis visualisiert, wobei die Farbe der Kreissegmente die Verkehrsauslastung darstellt \parencite[S. 2973 - 2974]{survey_traffic_data_visualization_2015}.

\begin{figure}[H]
    \caption{Radiale Visualisierung \parencite[S. 5]{radial_layout_t_watcher}}
    \includegraphics[width=.5\linewidth]{content/00_assets/radial_layout.png}
    \label{fig_radial_layout}
\end{figure}

\subsection{Räumliche Visualisierungen}
Räumliche Eigenschaften lassen sich gemäss Chen et al. in \textbf{punktebasierte, linienbasierte}, sowie \textbf{regionenbasierte} Visualisierungen einteilen \parencite[S. 2974 - 2975]{survey_traffic_data_visualization_2015}.

\textbf{Punktbasierte Visualisierungen} repräsentieren die Daten als Punkte. Über die visuellen Variablen Grösse und Farbe können dem Nutzer weitere Aspekte vermittelt werden. Der Vorteil von punktbasierten Visualisierungen besteht darin, dass der Nutzer den Zustand aller Objekte gleichzeitig beobachten kann. Zudem können Punktvisualisierungen dabei helfen, Cluster hervorzuheben. Jedoch besteht bei grossen Datenmengen die Gefahr von Visual Clutter. Eine Möglichkeit, um Visual Clutter zu vermindern, ist die Verwendung von Heatmaps in Kombination mit dem \textbf{\acrfull{kde}} Algorithmus (siehe Abbildung \ref{fig_heatmap_kde}).

\begin{figure}[H]
    \caption{Heatmap-Visualisierung mit KDE-Algorithmus \parencite{vait_system}}
    \includegraphics[width=.5\linewidth]{content/00_assets/heatmap_kde.png}
    \label{fig_heatmap_kde}
\end{figure}

\textbf{Linienbasierte Visualisierungen} werden eingesetzt, um den Fluss von Verkehrsnetzwerken zu veranschaulichen. Im Falle von Zugnetzwerken können etwa die Streckennetze als einzelne Linien (Trajektorien) dargestellt werden, wobei die Dicke der Linie die Auslastung und die Farbe die Verspätung visualisieren könnte. Bei einer grossen Anzahl von Linien entsteht jedoch auch wieder die Gefahr von Visual Clutter. Ein bewährter Lösungsansatz für dieses Problem ist das sogenannte «Edge Bundling». Hierbei werden ähnliche Linien zu einem «Bündel» zusammengefasst \parencite[S. 2974 - 2976]{survey_traffic_data_visualization_2015}. Eine weitere Möglichkeit besteht darin, anstelle von Edge Bundling den \acrshort{kde} Algorithmus auf die Trajektorien anzuwenden (siehe Abbildung \ref{fig_line_kde}).

\begin{figure}[H]
    \caption{Linienbasierte Visualisierung mit KDE-Algorithmus \parencite[S. 7]{streaming_data_kde}}
    \includegraphics[width=.5\linewidth]{content/00_assets/line_visualization_kde.png}
    \label{fig_line_kde}
\end{figure}

\textbf{Regionalbasierte Visualisierungen} erlauben es gemäss Chen et al. Makro-Patterns (Zugverspätungen pro Region) in Verkehrsdaten zu visualisieren. Sie sind jedoch nicht geeignet, um Mikro-Patterns von einzelnen Zügen zu visualisieren \parencite[S. 2976]{survey_traffic_data_visualization_2015}.

\section{Erkennung von visuellen Mustern}
Nebst den bereits besprochenen Methoden wie Edge Bundling und dem \acrshort{kde} Algorithmus existieren noch andere Möglichkeiten, um visuelle Muster eines Verkehrsdatensatzes hervorzuheben. Eine bekannte Vorgehensweise aus dem Bereich Data-Mining, um visuelle Muster zu identifizieren, ist das Clustering. Lopez et al. untersuchten während 35 Tagen die Verkehrsdaten von Amsterdam. Mithilfe des K-Means, DBSCAN sowie weiterer Cluster-Algorithmen konnten sie 9 verschiedene Cluster für das Verkehrsverhalten identifizieren \parencite{lopez_2017}. Toshniwal et al. nutzten ebenfalls verschiedene Clusteralgorithmen, um Muster im Stadtverkehr von Aarthus (Dänemark) zu identifizieren. Mithilfe von Sensoren an spezifischen Ortschaften wurde die Anzahl der vorbeifahrenden Fahrzeuge ermittelt. Die Daten wurden anschliessend entsprechend aggregiert (Minuten, Stunden etc.). Für die Analyse des zeitlichen als auch räumlichen Verhaltens wurden diverse Cluster-Algorithmen verwendet. Für das \textbf{zeitliche} Verhalten wurde der DBSCAN Algorithmus verwendet. Für das \textbf{räumliche} Verhalten wurde unter anderem ein hierarchisches Clustering (Agglomerative Nesting) verwendet (siehe Abbildung \ref{fig_clustering_urban_traffic}).

\begin{figure}[H]
    \caption{Cluster-Algorithmen gemäss Toshinwal et al. \parencite[S. 1050]{clustering_urban_traffic_data}}
    \includegraphics[width=.5\linewidth]{content/00_assets/urban_traffic_clustering_methods.png}
    \label{fig_clustering_urban_traffic}
\end{figure}

\section{Weitere Ansätze}
Selbstverständlich gibt es noch viele weitere Möglichkeiten, um Verspätungen innerhalb eines Netzwerks zu visualisieren, bspw. mithilfe von \textit{Simulationen} und \textit{Machine Learning} Verfahren. Xia et al. nutzten Deep Learning, um Verspätungen innerhalb des Zugnetzwerkes von Tokyo sowohl zu simulieren als auch vorherzusagen \parencite[S.56]{xia_2018}.

